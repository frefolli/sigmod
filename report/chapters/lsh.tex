\chapter{LSH}
L'approccio vincente si è basato su locality-sensitive hashing (LSH).
Gli approcci basati su LSH consistono nella definizione di una funzione di hash 
che ha l'obiettivo di partizionare il dataset $D$ in una serie di bucket, ciascuno 
formato in modo tale da contenere vettori che sono i più simili possibili, rispetto 
alla distanza euclidea. 

La costruzione della funzione di hash deve essere fatta in modo tale da 
massimizzare il numero di collisioni tra i vettori simili e ridurre al minimo 
il numero di collisioni per vettori differenti. 

In questo modo la ricerca si articola in:
\begin{itemize}
    \item calcolare l'hash della query e si ottiene l'indice del bucket
    \item calcolare le distanze tra la query e tutti i vettori nel bucket ed 
    estrarre i 100 più vicini
\end{itemize} 

Questo approccio permette di ridurre il numero di distanze calcolate, dal momento 
che si confrontano solo la query con i vettori nel bucket. Il problema è trovare 
un approccio per costruire una funzione di hash che sia veloce da calcolare e 
che permetta di creare dei bucket di buona qualità.

La complessità maggiore di questo approccio è quella di generare una funzione di 
hash che sia efficiente e che suddivida meglio i vettori nel dataset.