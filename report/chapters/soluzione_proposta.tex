\chapter{Soluzione proposta}
L'approccio vincente si è basato su locality-sensitive hashing (LSH).
Gli approcci basati su LSH consistono nella definizione di una funzione di hash 
che ha l'obiettivo di partizionare il dataset $D$ in una serie di bucket, ciascuno 
formato in modo tale da contenere vettori che sono i più simili possibili, rispetto 
alla distanza euclidea. 

La costruzione della funzione di hash deve essere fatta in modo tale da 
massimizzare il numero di collisioni tra i vettori simili e ridurre al minimo 
il numero di collisioni per vettori differenti. 

In questo modo la ricerca si articola in:
\begin{itemize}
    \item calcolare l'hash della query e si ottiene l'indice del bucket
    \item calcolare le distanze tra la query e tutti i vettori nel bucket ed 
    estrarre i 100 più vicini
\end{itemize} 

Questo approccio permette di ridurre il numero di distanze calcolate, dal momento 
che si confrontano solo la query con i vettori nel bucket. Il problema è trovare 
un approccio per costruire una funzione di hash che sia veloce da calcolare e 
che permetta di creare dei bucket di buona qualità.

\section{Costruzione della funzione di hash}

La funzione di hash che è stata definita si basa su \textbf{Random Projection}(RP),
esattamente lo stesso algoritmo di riduzione di dimensionalità. 
L'idea è di definire $k' = \lceil \log_2{\sqrt{|D|}} + 2\rceil$ iperpiani randomici che suddividono
lo spazio dei vettori del dataset in $2^{k'}$ regioni, ad ogni regione si associa 
un hash e tutti i vettori che cadono nella stessa regione verranno mappati sullo 
stesso hash, quindi inseriti nello stesso bucket.

\begin{nota}
La scelta del valore $k'$ è stata effettuata con una gridsearch degli iperparamentri 
che sarà illustrata nelle sezioni successive.
\end{nota}

Ogni singolo $i$-esimo iperpiano è stato definito dal suo vettore norma $n_i$
generato randomicamente, estraendo le singole componenti da una distribuzione uniforme 
($n_i \in U[-1,1]^{100}$). L'hash di un vettore viene calcolato considerandolo 
come un numero binario di $k'$ bit, dove l'$i$-esimo bit $h_i$ dell'hash specifica:
\begin{itemize}
    \item $h_i = 1$: il vettore $v$ è posizionato sopra L'iperpiano definito dalla 
    norma $n_i$
    \item $h_i = 0$: il vettore $v$ è posizionato sotto L'iperpiano definito dalla 
    norma $n_i$
\end{itemize}

A livello matematico si riduce all'equazione 
$$h_i = \begin{cases}
    1 & n_i \cdot v \ge 0\\
    0 & n_i \cdot v < 0
\end{cases}$$ 

Dove $\cdot$ è il prodotto interno in $\mathbb{R}^{100}$. Lo pseudocodice della 
funzione di hash è stato riportato nel listato \ref{list:funzione_hash}.

\begin{algorithm}
    \SetAlgoLined
    \KwData{vettore $v\in D$}
    \KwResult{hash $h\in \mathbb{N}$}

    \Begin{
        $h= 0$\;
        \For{$i=1$ \KwTo $k'$}{
            $partial = n_i \cdot v$\;
            $h = h << 1$\;
            \If{$partial \ge 0$}{
                $h= h+1$\;
            }
        }
    }
    \caption{Pseudocodice della funzione di hash}
    \label{list:funzione_hash}
\end{algorithm}

\begin{nota}
    L'hash viene calcolato solo sulle componenti del vettore e non sui metadati.
\end{nota}

La complessità della funzione di hash è $\mathcal{O}(k'\cdot n)$

\section{Costruzione della hash table}
