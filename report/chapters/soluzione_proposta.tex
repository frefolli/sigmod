\chapter{Soluzione proposta}
L'approccio vincente si è basato su \textbf{locality-sensitive hashing} (LSH).
Gli approcci basati su LSH consistono nella definizione di una funzione di hash 
che ha l'obiettivo di partizionare il dataset $D$ in una hash table composta 
da una serie di bucket, ciascuno 
formato in modo tale da contenere vettori che sono i più simili possibili, rispetto 
alla distanza euclidea. 

La costruzione della funzione di hash deve essere fatta in modo tale da 
massimizzare il numero di collisioni tra i vettori simili e ridurre al minimo 
il numero di collisioni per vettori differenti. 

In questo modo la ricerca si articola in:
\begin{itemize}
    \item calcolare l'hash della query che definisce l'indice del bucket
    \item calcolare le distanze tra la query e tutti i vettori nel bucket ed 
    estrarre i 100 più vicini
\end{itemize} 

Questo approccio permette di ridurre il numero di distanze calcolate, dal momento 
che si confrontano solo la query con i vettori nel bucket. Il problema è trovare 
un approccio per costruire una funzione di hash che sia veloce da calcolare e 
che permetta di creare dei bucket di buona qualità.

\section{Costruzione della funzione di hash}

La funzione di hash che è stata definita si basa su \textbf{Random Projection}(RP),
esattamente lo stesso algoritmo di riduzione di dimensionalità. 

L'idea è di definire $k' = \lceil \log_2{\sqrt{|D|}} + 2\rceil$ iperpiani randomici che suddividono
lo spazio dei vettori del dataset in $2^{k'}$ regioni, ad ogni regione si associa 
un hash e tutti i vettori che cadono afferiscono alla stessa regione produrranno 
una collisione, quindi verranno inseriti nello stesso bucket.

\begin{nota}
La scelta del valore $k'$ è stata effettuata con una gridsearch degli iperparamentri 
che sarà illustrata nelle sezioni successive.
\end{nota}

Ogni singolo $i$-esimo iperpiano è stato definito dal suo vettore norma $n_i$,
generato randomicamente estraendo le singole componenti da una distribuzione uniforme 
($n_i \in U[-1,1]^{100}$). L'hash di un vettore viene calcolato considerandolo 
come un numero binario di $k'$ bit, dove l'$i$-esimo bit $h_i$ dell'hash specifica:
\begin{itemize}
    \item $h_i = 1$: il vettore $v$ si trova sopra L'iperpiano definito dalla 
    norma $n_i$
    \item $h_i = 0$: il vettore $v$ si trova sotto L'iperpiano definito dalla 
    norma $n_i$
\end{itemize}

A livello matematico si riduce all'equazione 
$$h_i = \begin{cases}
    1 & n_i \cdot v \ge 0\\
    0 & n_i \cdot v < 0
\end{cases}$$ 

Dove $\cdot$ è il prodotto interno in $\mathbb{R}^{100}$. Lo pseudocodice della 
funzione di hash è stato riportato nel listato \ref{list:funzione_hash}.

\begin{algorithm}
    \SetAlgoLined
    \KwData{vettore $v\in D$}
    \KwResult{hash $h\in \mathbb{N}$}

    \Begin{
        $h= 0$\;
        \For{$i=1$ \KwTo $k'$}{
            $partial = n_i \cdot v$\;
            $h = h << 1$\;
            \If{$partial \ge 0$}{
                $h= h+1$\;
            }
        }
    }
    \caption{Pseudocodice della funzione di hash}
    \label{list:funzione_hash}
\end{algorithm}

\begin{nota}
    L'hash viene calcolato solo sulle componenti del vettore e non sui metadati.
\end{nota}

La complessità temporale della funzione di hash calcolata un un particolare $v\in V$,
è $\mathcal{O}(k' \cdot |v|)$ dove 
$|v|$ è il numero delle componenti di $v\in D$, dai requisiti della challenge 
costante a $100$, perciò si può anche approssimare a $\mathcal{O}(k')$ che, dal momento 
che anche $k'$ è costante e piccolo allora si più approssimare a $\mathcal{O}(1)$.

\section{Costruzione di LSH}

La struttura dati deve contenere tutti i bucket e il mapping, per comodità bidirezionale,
ciascun vettore $v\in D$ con il suo hash associato.

La costruzione della struttura si articola in tre passi:
\begin{itemize}
    \item calcolo dell'hash per ogni vettore del dataset
    \item creazione dei bucket e inserimento dei vettori
    \item ordinamento dei vettori nei bucket
\end{itemize}

Il calcolo dell'hash viene effettuato per ogni vettore e si salva il risultato 
all'interno di un array $v_h$:

$$v_h[i] = h(v_i)$$

Successivamente si costruiscono i bucket $M_h$, a livello implementativo ciascun bucket 
è una \texttt{unordered\_map}, che coincide con 

$$M_h = \{v\in D : h = h(v)\}$$

\begin{nota}
    La scelta di utilizzare \texttt{unordered\_map} piuttosto che \texttt{map} è dovuta ai tempi di 
    accesso e ricerca degli elementi:
    \begin{itemize}
        \item \texttt{unordered\_map}: il tempo di accesso usando l'hash è costante 
        nel caso medio e lineare nel caso peggiore, inoltre il tempo di ottenimento dell'hash 
        dato un vettore è costante nel caso medio e lineare nel caso peggiore. \cite{unsorted_map_index}\cite{unsorted_map_find}
        \item \texttt{map}: il tmepo di accesso usando l'hash è logaritmico nella dimensione 
        del container, mentre il tempo di ottenimento dell'hash dato un vettore è
        logaritmico. \cite{map_index}\cite{map_find}
    \end{itemize}
\end{nota}

Per ultimo si effettua un ordinamento dei vettori nei singoli bucket, più precisamente
si ordina prima per categoria, timestamp e successivamente per le componenti dei vettori.

I tempi di costruzione della struttura coincidono con $\mathcal{O}(|D|\cdot 1)$
per il mapping vettore-hash, $\mathcal{O}(|D|\cdot |D|)$ ($\Theta(|D|)$) per l'inserimento 
dei vettori nei bucket e $\mathcal{O}(|D|\log |D|)$ per l'ordinamento dei vettori 
nei singoli bucket, in totale ne caso peggiore si ha $\mathcal{O}(|D|\cdot |D|)$.

Parlando invece dello spazio occupato, la struttura ha una complessità spaziale 
limitata perché non si salvano completamente i vettori in ciascuna sottostuttura,
bensì si salvano i loro indici di posizione del dataset, in aggiunta gli hash non 
sono altro che interi. Quindi il vettore di mapping vettori-hash $v_h$ alla fine 
contiene un totale $|D|$ interi, mentre l'unsorted map conterrà sempre gli indici 
dei vettori, in totale $|D|$, e tutti gli hash, quindi al massimo $2^{k'} = 2^ {14}$ 
interi, infine complessivamente si ha un totale di al massimo $10^7 + 2^14$ interi 
da salvare ($\approx 38 Mb$).

Il problema di questa soluzione è la Recall che è risultata bassa fin da subito, 
in fatti uno dei problemi e allo stesso tempo pregi di questo metodo è che la ricerca 
dei vettori più vicini si effettua solo nel bucket in cui cade il vettore di query.
Questo permette di ridurre il numero di distanze calcolate per ogni query, ma non 
considera anche vettori vicini alla query che fanno parte di bucket adiacenti.

\begin{esempio}
    Se la query è di tipo $1$, quindi si deve filtrare per categoria, c'è il rischio 
    che in un particolare bucket non ci siano abbastanza vettori della stessa categoria
\end{esempio}

\begin{esempio}
    Possiamo aver problemi anche nel caso in cui la query sia di tipo $0$, quindi si 
    deve filtrare per metadati, e cade nello spazio vicino ad un iperpiano che limita il bucket, 
    questo comporta che in fase di ricerca si consideranno solo i vettori di quel 
    bucket e non quelli del bucket adiacente che ha in comune lo stesso iperpiano. 
\end{esempio}

Per risolvere i problemi è stato pensato di generare nuovi mapping randomici (\textbf{LSH con hash table multiple})

\section{LSH con hash table multiple}

Precedentemente è stato introdotto LSH composta da una hash table che mappa i vettori 
in bucket identificati da un hash.

L'idea di LSH con hash table multiple è di creare un totale di $M$ hash table ognuna con un
inizializzazione tramite iperpiani il più differente possibile dalle altre. In questo 
modo è possibile suddividere l'intero spazio dei vettori in $M$ modi differenti,
cosicché in fase di ricerca, al posto che cercare in una hash table, si cerca 
in $M$ hash table parallelamente. In questo modo si può risolvere il problema di 
cercare i vettori vicini anche nei bucket adiacenti, perché si spera che i bucket 
in cui cade la query per ciascuna hashtable sia il più differente possibile dagli 
altri. $M$ è stato scelto attraverso una gridsearch che verrà illustrata successivamente.

La differenza di partizione dello spazio di ciascuna hashtable dipende interamente 
da come vengono generati gli iperpiani di separazione, dal momento che si estraggono 
da una distribuzione uniforme si ha già una buona variabilità e questo permette di avere 
una maggior variabilità anche nelle hash table, portando a dei miglioramenti in termini di 
Recall.

Bisogna però menzionare il fatto che le diverse hash table partizionano lo stesso 
spazio geometrico, quindi in fase di ricerca può capitare che la query venga confrontata 
più volte con gli stessi vettori perché questi, ess endo vicini, possono capitare 
negli stessi bucket della query in hashtable differenti.

Visto che l'introduzione di hash table multiple può portare a confronti multipli 
tra la query e gli stessi vettori, per ridurre i tempi di ricerca e aumentare 
la recall sulle query di tipo $1$ e di tipo $3$ è stato pensato di risolverlo 
definendo \textbf{LSH Forest}.

\section{LSH Forest}
L'approccio di LSH Forest è di velocizzare le query che filtrano sulla categoria,
ovvero quelle di tipo $1$ e $3$. Per fare ciò è stato pensato di definire un LSH
con hash table multiple generale sull'intero dataset per le query di tipo $0$ e $2$, 
inoltre una LSH con hash table multiple per ogni categoria che contiene almeno $2000$ 
vettori, in questo modo si riduce il calcolo delle distanze. La scelta di creare 
una LSH dedicata solo per le categorie che superano la cardinalità dei $2000$ vettori 
è dovuta dal fatto che il tempo necessario per crearla è maggiore dei benefici 
che porta nella ricerca, quindi si preferisce evitare di crearla.

L'unico problema che rimane da risolvere in fase di costruzione di LSH Forest è 
indentificare i vettori della stessa categoria nel dataset, per fare ciò è bastato 
ordinare il dataset dei vettori secondo la categoria, il timestamp ed infine 
le componenti di ciascun vettore. Questo richiede $\mathcal{O}(|D|\log |D|)$ di 
complessità, ma non aggiunge complessità alla costruzione delle LSH per ogni categoria 
dal momento che i vettori della stessa categoria sono tutti continui. Questa 
fase di ordinamento velocizza l'ordinamento dei vettori nei bucket e la ricerca 
per filtrando per il timestamp. L'ordinamento dei vettori del bucket non è detto 
che venga preservato perché la costruzione è parallelizzata, perciò l'ordinamento 
non è assicurato.

L'algoritmo finale di costruzione di LSH Forest è mostrato nel listato

\begin{algorithm}
    \SetAlgoLined
    \KwData{dataset $D$}
    \KwResult{LSH Forest $LSHF$}

    \Begin{
        $sort(D)$\;
        
        \For{$i=1$ \KwTo $k'$}{
            $partial = n_i \cdot v$\;
            $h = h << 1$\;
            \If{$partial \ge 0$}{
                $h= h+1$\;
            }
        }
    }
    \caption{Pseudocodice della costruzione di LSH Forest}
    \label{list:costruzione_lsh_forest}
\end{algorithm}

\section{Ricerca in LSH Forest}
La ricerca nella struttura cambia in base alla tipologia di query che si deve 
cercare.