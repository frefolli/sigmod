\chapter{Soluzione vincente}
La soluzione vincente si basa su grafi metrici.
Più precisamente crea una struttura di indicizzazione a grafo ibrida, questa struttura 
è composta da un insieme di grafi HNSW, ciascuno contenente un insieme di vettori 
sufficientemente grande con metadati comuni. Ad esempio: crea un HNSW per ogni singola 
categoria sufficientemente grande e un HNSW per intervallo di timestamp.

\section{HNSW teorico}
Hierarchical Navigable Small World Graphs (HNSW) è una struttura dati basata su 
grafi metrici a livelli che permette di navigare
lo spazio dei vettori attravero un cammino composto da archi che collegano due nodi 
vicini. Ogni livello è una 
rappresentazione approssimata dello spazio gentrico tramite un grafo metrico. Nel grafo ogni nodo rappresenta 
un vettore e si hanno degli archi diretti che rappresentano la relazione di vicinanza,
ovvero, dato un nodo ogni suo arco uscente porta ai $M$ nodi più vicini a quel nodo. La relazione 
di vicinanza è diretta perché nel caso dell'arco $(a,b)$, non è detto che il nodo 
$b$ abbia nei $M$ nodi più vicini il nodo $a$.

Ogni livello rappresenta il livello di approssimazione dello spazio geometrico, 
più il livello è alto allora maggiore è l'approssimazione, più il livello è basso 
e più precisa sarà la rappresentazione dello spazio. Dato un livello $l$, il livello 
successivo $l-1$ è un grafo composto da almeno gli stessi nodi del grafo sul livello 
precedente con l'aggiunta di nuovi nodi collegati ai nodi già presenti secondo 
la relazione di vicinanza (nel nostro caso si basa sulla distanza euclidea, Vedi 
figura \ref{fig:hnsw}). 

Questo permette di navigare lo spazio velocemente nel livello più alto non passando 
per tutti i punti perché ogni nodo di un livello rappresenta una \textbf{Voronoi Cell} dello spazio grande 
quanto la specificità del livello stesso. Nodi presenti in un grafo di livello 
superiore rappresenteranno delle Voronoi Cell più grandi rispetto ai nodi 
presenti in un grafo di un livello inferiore.

\begin{figure}
    \centering
    \includegraphics*[width=0.5\textwidth]{img/hnsw.png}
    \caption{Esempio grafico di un HNSW a $3$ livelli. Il percorso con le frecce blu corrisponde 
    al percorso compiuto per cercare il nodo giallo nella struttura. 
    Il livello $0$ rappresenta l'intero spazio metrico, mentre i livelli superiori sono delle 
    approssimazioni. \cite{hnsw}}
    \label{fig:hnsw}
\end{figure}

\subsection{Creazione di HNSW}

Dato un insieme di vettori $V$, la costruzione di un generico HNSW  è un processo 
iterativo, in cui si inserisce ciascun vettore uno alla volta. L'inserimento di 
un vettore $q\in V$ nella struttura si articola nelle seguenti fasi:
\begin{itemize}
    \item \textbf{inizializzazione}: si seglie un nodo $ep$ di entry point nel livello massimo
    e si seleziona il livello $l$ a partire dal quale si inserice il vettore $q$.
    Più precisamente $l = \lfloor-\ln (U[0,1])\cdot m_l\rfloor$  e si inserirà $q$ 
    in tutti i livelli da $l$ a $0$. $m_l$ è la costante di normalizzazione che 
    limita $l\in [0, L_{max}]$, dove $L_{max}$ è il livello massimo della struttura.
    \begin{figure}
        \centering
        \includegraphics*[width=0.5\textwidth]{img/exp_decay_distr.png}
        \caption{Andamento di una funzione di densità di decadimento esponenziale. 
        \cite{exp_decay}}
        \label{fig:exp_decay}
    \end{figure}
    \item \textbf{prima fase}: a partire dal livello $L_{max}$, si naviga il grafo 
    a partire dal nodo $ep$ fino a quando non si trova il nodo $w$ più vicino a
    $q$ (algoritmo greedy), successivamente si effettua la stessa operazione 
    su livello $L_{max} - 1$, utilizzando come nodo di entry point il nodo $w$,
    si itera il procedimento fino a quando non si trova l'entry point $w$ del livello $l$.
    \item \textbf{seconda fase}: a partire dal livello $l$ si cercano i $W$ nodi 
    più vicini a $q$ a partire dall'ultimo entry point trovato al termine della fase precedente ($|W| = efConstruction$),
    sempre col medesimo algoritmo greedy accennato nella prima fase.
    Successivamente si usa un'euristica che calcola dall'insieme dei $W$ nodi
    l'insieme $N$ degli $M$ nodi più vicini a $q$ sempre sullo stesso layer $l$. 
    Si creano degli edge bidirezionali tra i nodi $N$ e $q$. Successivamente si 
    sistemano i collegamenti tra i vicini in modo che ogni nodo rispetti il vincolo 
    di avere al massimo $M_{max}$ collegamenti, questo viene implementato rimuovendo 
    degli edge tra due nodi nel vicinato di $q$. Infine si reitera il procedimento 
    dal livello $l$ al livello $0$.
\end{itemize}

\begin{nota}
    La scelta del livello dal quale inserire un nuovo vettore viene effettuata 
    in modo randomico secondo una \textbf{distribuzione di decadimento esponenziale}. 
    In sostanza si ha maggior probabilità di estrarre un livello $l$ vicino a $0$ 
    piuttosto che un livello più vicino a $L_{max}$. (vedi figura \cite{exp_decay})
\end{nota}

\begin{nota}
    L'algoritmo greedy accennato precedentemente permette di navigare il grafo 
    ad un particolare livello a partire da un nodo di entrypoint $ep$ e trovare gli $ef$ 
    nodi candidati vicini al vettore $q$. 
    La ricerca avviene nel seguente modo:
    \begin{itemize}
        \item si tiene traccia dei nodi già visitati, dei nodi vicini a $q$. Inizialimente 
        entrambi gli insiemi sono vuoti.
        \item si parte dal nodo $ep$, si aggiunge $ep$ all'insieme dei nodi vicini a $q$,
        si guardano tutti i vicini e si calcola la distanza da ciascun vicino a $q$,
        si sceglie il vicino a distanza minima e ci si sposta in quel nodo, si 
        aggiunge $ep$ all'insieme dei nodi visitati.
        \item si reitera il punto precedente, evitando di visitare nodi già visitati e
        fino a quando non si può più andare avanti, oppure si continua fino a quando 
        non si arriva ad una iterazione massima. Se l'insieme dei vicini a $q$
        superla $ef$ allora vengono rimossi i nodi più lontani da $q$.
    \end{itemize} 
\end{nota}

Per la costruzione della struttura si richiedono i seguenti parametri:
\begin{itemize}
    \item $M_{max}$: numero massimo di edge uscenti da ogni nodo per un livello, questo 
    potrebbe coincidere con $M$.
    \item $L_{max}$: livello massimo della struttura.
    \item $efConstruction$: specifica il numero dei nodi candidati ad essere i vicini 
    del vettore $q$. Questo parametro permette di variare la recall e il tempo di 
    costruzione della struttura.
\end{itemize} 



\subsection{Ricerca nella struttura}
La ricerca di una query $q$ in un generico HNSW richiede i seguenti parametri:
\begin{itemize}
    \item $K$: numero di vicini alla query $q$ da cercare nella struttura
    \item $ef$: numero di candidati vicini da considerare quando si naviga sui singoli 
    grafi di ciascun livello.
\end{itemize}

La ricerca si articola nei seguenti passi:
\begin{itemize}
    \item si sceglie randomicamente un entry point $ep$ nel livello massimo
    \item nel livello massimo si cercano gli $ef$ migliori vicini alla query 
    $q$ mediante l'algoritmo greedy di graph traversal usato in fase di costruzione
    \item si estrae dai migliori $ef$ vicini alla query il più vicino e lo si usa 
    come $ep$ per il grafo al livello inferiore.
    \item si reitera il procedimento fino a quando non si ottengono gli $ef$ vicini 
    alla query al livello $0$. In questo caso si ritornano i migliori $K$ vicini.
\end{itemize}  

\section{Creazione della struttura a grafo ibrida}
Per prima cosa vene letto tutto il dataset e viene salvato
all'interno di $3$ array:
\begin{itemize}
    \item \texttt{base\_vecs}: array contenente i coefficienti di tutti i vettori del dataset
    di dimensione $10^7\cdot 100$.
    \item \texttt{base\_labels}: array contente le labels di tutti i vettori, ovvero un 
    vettore di $10^7$ elementi.
    \item \texttt{base\_timestamps}: array contente i timestamps di tutti i vettori, ovvero un 
    vettore di $10^7$ elementi.
\end{itemize} 

\begin{nota}
    La posizione $i$ indica  l'informazione dell'$i$-esimo vettore nel dataset.
\end{nota}

Successivamente si creano degli array di appoggio che specificano l'ordinamento 
del dataset rispetto a diversi criteri di ordinamento:
\begin{itemize}
    \item \texttt{sorted\_base\_ids}: vettore contenente gli indici dei vettori ordinati 
    per categoria crescente e successivamente per timestamps. Quindi per ogni $i\le j\in \mathbb{N}$,
    allora il numero della categoria associato al vettore $sorted\_base\_ids[i]$
    è minore rispetto al numero della categoria del vettore $sorted\_base\_ids[j]$,
    in caso di categorie uguali allora si ordina per timestamps.
    \item \texttt{sorted\_base\_ids\_by\_time}: vettore contenente gli indici dei vettori ordinati 
    per timestamps spostando i spostando tutti i vettori della categoria più grande all'inizio 
    del vettore. In questo modo all'inizio del vettore si avranno solo vettori 
    ordinati per timestamps della categoria più grande, successivamente si hanno 
    tutti gli altri vettori ordinati per timestamp. 
    \item \texttt{sorted\_base\_ids\_by\_full\_time}: vettore contenente gli indici dei vettori ordinati 
    per timestamps.
\end{itemize}

\begin{nota}
    L'ordinamento dei vettori della categoria più grande in sorted\_base\_ids\_by\_time 
    viene replicato dal vettore sorted\_base\_ids perché in caso ci fossero vettori 
    della categoria più grande con lo stesso timestamp allora questi possono avere 
    un ordinamento diverso tra i due indici.
\end{nota}

In aggiunta viene creata una $c\_map$, ovvero una mappa delle categorie ovvero 
un array di coppie ordinate:
\begin{equation}
    category\_map[c] = \langle i, dimensione\_categoria\_c\rangle
\end{equation} 
dove $i = sorted\_base\_ids[v]$ con $v$ ultimo vettore della categoria $c$, 
mentre $dimensione\_categoria\_c$ specifica il numero di vettori presenti nella 
categoria $c$. Da notare che la coppia viene aggiunta all'array solo quando $c$ 
ha almeno $450000$ vettori.

Successivamente si costruisce una $t\_map$, ovvero una mappa dei timestamp ovvero 
un array di coppie ordinate:
\begin{equation}
    t\_map[i] = \langle l, dimensione\_intervallo\rangle
\end{equation} 
dove:
\begin{itemize}
    \item $i\in \{0,\dots,9\}$ specifica l'intervallo di timestamp $[0.1*i, 0.1*(i+1)]$
    \item $l= timestamps\_by\_full\_time[v]$, $v$ è il primo vettore con il timestamp nell'intervallo cercato
    \item $dimensione\_intervallo$ il numero di vettori nell'intervallo
\end{itemize}

In seguito viene letto il query set e viene salvato all'interno di due array:
\begin{itemize}
    \item \texttt{query\_vecs}: array contente le componenti dei vettori di query ovvero 
    $10^6 \cdot 100$.
    \item \texttt{query\_metas}: array contente i metadati delle query, ovvero la tipologia,
    la categoria e l'intervallo di timestamp.
    $10^6 \cdot 4$.
\end{itemize}

Successivamente si creano degli array di appoggio:
\begin{itemize}
    \item \texttt{sorted\_ids}: vettore degli indici dei delle query ordinate per tipo, 
    categoria, lower bound e upper bound dell'intervallo di timestamp
    \item \texttt{category\_query}: mappa che associa per ogni categoria un vettore di 
    indici di query che afferiscono a quella categoria.
\end{itemize}

Da notare che la $c\_map$, $category\_query$ e $t\_map$ sono tutte mappe implementate 
come \texttt{unordered\_map}.

Dopo la lettura ordinata del dataset e del queryset, è stato applicato un passo di 
\textbf{quantizzazione scalare simmetrica} in modo da ridurre lo spazio occupato dalle componenti, 
più precisamente si riduce da $400B$ a $112B$, trasformando tutte le combonenti in 
un byte. Ogni vettore occupa $112B$ perché $100B$ sono per le $100$ componenti mentre i restanti $12B$
sono posti a zero e servono ad un duplice scopo: avere un'allineamento a $32bits$/$64bits$ per questioni
di indirizzamento (altrimenti non ci sono assicurazioni su come compilatore intende allineare la memoria)
e avere contemporaneamente un allineamento a $16B$ per poter sfruttare le operazioni del \textbf{SSE} (estensione SIMD
che permette di operare in parallelo su registri a $128bits$).

In senguito è stato creato un HSNW per i vettori 
afferenti alla categoria più grande, uno per i vettori afferenti a categorie che 
contengono almento $500000$ vettori e uno per ogni intervallo di timestamp della 
$t\_map$. Gli HNSW vengono costruiti sulle codifiche dei vettori.

In aggiunta, vengono calcolate tutte le codifiche della quantizzazione scalare 
di tutti i vettori del dataset. Inoltre, si creano due vettori di ordinamento 
che specificano l'ordinamento delle condifiche:
\begin{itemize}
    \item \texttt{sorted\_both}: in posizione $i$ contiene i metadati dell'$i$-esima
    codifica dei vettori del dataset secondo l'ordinamento per categoria e successivamente
    per timestamp.
    \item \texttt{sorted\_timestamp}: in posizione $i$ contiene i metadati dell'$i$-esima
    codifica dei vettori del dataset secondo l'ordinamento per timestamp.
\end{itemize}

%La struttura di indicizzazione definita dagli autori è composta da:
%\begin{itemize}
%    \item \texttt{base\_vecs}
%    \item \texttt{base\_labels}
%    \item \texttt{sorted\_base\_ids}
%    \item \texttt{sorted\_base\_ids\_by\_time} 
%    \item \texttt{sorted\_base\_ids\_by\_full\_time}
%    \item \texttt{sorted\_both}
%    \item \texttt{sorted\_timestamp}
%    \item \texttt{c\_map}
%    \item \texttt{t\_map}
%    \item un grafo HNSW per la categoria più grande.
%    \item un grafo HNSW per ogni categoria più grande di $500000$ vettori.
%    \item un grafo HNSW per ogni intervallo di timestamp, in totale $10$.
%\end{itemize}

\subsection{Quantizzazione Scalare Simmetrica}

Per poter ridurre i dati di dimensionalit\`a si hanno due passi:
\begin{itemize}
    \item \textbf{training}: si allena l'algoritmo per definire gli iperparametri
    \item \textbf{codifica}: si codificano i vettori usando i parametri trovati nella 
    fase precedente.
\end{itemize}

La fase di training si esegue su un sottoinsieme di vettori del dataset e permette 
di definire l'unico iperparametro $\alpha$. L'iperparametro viene definito come 
il valore assoluto del coefficiente in valore assoluto maggiore di tutti i coefficienti 
dei vettori del training set.

In seguito si codificano tutti i vettori del dataset dividendo ogni coefficiente 
per $\alpha$ e nel caso i coefficienti "normalizzati" uscissero dall'intervallo 
$[-1,1]$ allora di riportano all'estremo più vicino (i valori eccedenti gli estremi 
vengono riassegnati all'estremo che superano, ex: $-5 \rightarrow -1$). Successivamente 
si moltiplica ciascun coefficiente per $127$ e si converte ad intero, in questo modo 
si ottengono $255$ valori possibili per ogni coefficiente nell'intervallo $[-127,127]$.

La quantizzazione \`e detta simmetrica perch\`e ogni float a $32bit$ viene compresso in un valore intero a $8bit$ nell'intervallo $[-127, 127]$.
Viceversa nella quantizzazione asimmetrica (implementata ma non utilizzata dagli autori) i valori sarebbero compressi nell'intervallo $[0, 255]$.

\subsection{Implementazione di HNSW}

\paragraph{Configurazione}

Vengono costruiti un grafo HNSW per la categoria $C_b$ pi\`u grande, uno per ogni categoria $C_i \ne C_b$ con $|C_i| > 500000$ e infine uno per ogni intervallo di timestamp nella $t\_map$.
I parametri di costruzione degli HNSW specificati dagli autori sono $M = 28$ e $efConstruction = 200$.
In generale il numero di massimo di layer ottenibili nell'HNSW scala come $\frac 1 {ln(M)}$, ovvero con una probabilit\`a
 molto piccola verr\`a estratto un layer superiore allo zero
 (si richiama la sezione d'introduzione su HNSW: la frazione \`e moltiplicata per una variabile uniforme $x \; : \; [0,1]$).
Con questa configurazione di parametri ci si pu\`o aspettare asintoticamente un numero molto basso di layer.

\paragraph{Il Wrapper}

Gli autori della soluzione hanno sfruttato l'implementazione di HNSW della libreria \href{https://github.com/zilliztech/pyglass/tree/master}{pyglass} creando un wrapper per le strutture definite dalla libreria.
Questo wrapper applica la quantizzazione scalare simmetrica a $8$ bit descritta precedentemente per sfruttare le operazioni vettoriali del processore, velocizzando il calcolo delle distanze.
Quindi aggiunge i punti sequenzialmente ma sfruttando la parallelizzazione: questo grazie alla presenza e uso di lock all'interno dell'implementazione dell'HNSW di \textit{pyglass}.
Il primo punto inserito \`e per\`o escluso dalla parallelizzazione, altrimenti eseguendo in parallelo verrebbero aggiungiunti fino a 32 nodi come entry point.
Dopo aver inserito tutti i dati, il wrapper costruisce un grafo che copia sostanzialmente il Layer 0 dell'HNSW, tracciando oltre agli indici dei nodi puntati dagli archi il timestamp di ogni nodo bersaglio.
Questo grafo contiene anche un \textbf{Initializer} utilizzato in fase di ricerca per inizializzare il pool di candidati nodi vicini.

\paragraph{La Libreria}

L'algoritmo di inserimento dei punti implementato dalla libreria ricalca essenzialmente gli algoritmi presentati nella sezione precedente e presentati dal paper ufficiale di HNSW, tuttavia applica una serie di ottimizzazioni che lo rendono pi\`u veloce di una implementazione naive.

Per cominciare, la fase iniziale di ricerca per l'entry point pi\`u vicino viene effettuata tramite una variante della ricerca su Layer che non tiene traccia di tutti i nodi identificati e che sono candidati ma solo del pi\`u vicino, quindi ogni iterazione all'interno di un livello tiene conto solo dell'euristicamente "migliore" vicino.
Questo ha l'effetto di velocizzare quella fase a discapito della precisione, ragion per cui nella fase successiva si tiene traccia di una coda di candidati per evitare di perdere informazioni.

Nella funzione di ricerca su layer vera e propria, tra gli accorgimenti troviamo delle istruzioni Streaming SIMD Extension che permettono all'hardware di precaricare le pagine con i dati dei nodi da usare nella Cache e la possibilit\`a di operare in parallelo sui dati quantizzati, il che velocizza i tempi di accesso e calcolo delle distanze in maniera sensibile.
A differenza del paper, nella fase di aggiornamento dei collegamenti in un layer si ritarda l'inserimento dell'arco dal vicino i-esimo al punto che si sta inserendo e si sfrutta l'euristica di selezione di migliori M vicini per ridurre gli archi del nodo vicino e solo successivamente aggiornare le liste di adiacenza (l'unica aggiornata immediatamente \`e quella del nodo in inserimento).

Infine il fatto di inserire parallelamente i punti permette ad ogni sottoricerca di considerare meno punti e questo snellisce di poco l'aggiunta di nuovi nodi al grafo.
Questo \`e possibile solo tramite l'utilizzo di un lock per ogni lista di adiacenza, un lock per la tabella degli identificatori interni all'HNSW e un lock "globale" per la definizione dell'entry point e del massimo livello nel grafo gerarchico.
Quando un thread deve leggere o scrivere dati da uno di questi punti, acquisisce il lock e lo mantiene fino al termine dell'operazione locale, quindi lo rilascia in modo che gli altri 31 thread possano procedere concorrentemente.
Euristicamente visto che c'\`e variabilit\`a sul vicinato del punto in inserimento e sul livello in cui si parte, la costruzione rimane concorrente invece che regredire a sequenziale.

\section{Ricerca nella struttura di indicizzazione}

La ricerca nella struttura a grafo ibrida ha diverse strategie in base alle singole 
query. In generale si effettua una ricerca sfruttando le codifiche dei vettori, 
in questo caso si estrae un insieme di vettori maggiore rispetto a quelli richiesti 
dalla challenge, in generale almeno $140$. Poi in un secondo momento si raffina 
il risultato andando ad estrare da questi $140$ i migliori $100$ usando la loro 
rappresentazione nello spazio originale non quantizzato. 

La prima operazione della ricerca si basa sull'ordinamento di tutte le query 
secondo due criteri differenti:
\begin{itemize}
    \item il primo per le label e poi per timestamp
    \item il secondo per timestamp
\end{itemize} 
Successivamente per ogni query viene calcolata la sua codifica usando la quantizzazione 
scalare sui coefficienti del vettore di query.

In seguito, viene calcolata la selettività di ciascuna query, per selettività
si intende il numero totale di vettori candidati ad essere potenziali vicini alla 
query rispetto i parametri di ricerca sui metadati. 
La selettività viene calcolata in tempo costante confrontando solo i metadati in 
base alla tipologia di query, sfruttando la $c\_map$ e la $t\_map$. 

La selettività di una query determina, insieme ai metadati della query, la strategia 
da adottare in fase di ricerca.

La \textbf{prima strategia} di ricerca è nel caso in cui la query è di tipo $1,2,3$ 
e la sua selettività è inferiore rispettivamente a $500000$, per le prime due tipologie, e $800000$, 
per l'ultimo tipo di query. In questo caso 
si effettua una ricerca esaustiva sfruttando le codifiche dei vettori candidati 
e gli array di ordinamento delle codifiche (\texttt{sorted\_both} e \texttt{sorted\_timestamp}).
Durante la ricerca si estraggono $140$ vettori vicini al vettore di query nello 
spazio delle condifiche e infine, di questi vettori, si scelgono $100$ vettori 
più vicini alla query nello spazio originale. 

La \textbf{seconda strategia} di ricerca è per le query di tipo $1$ che ha una selettività 
maggiore di $500000$ vettori. In questo caso la ricerca si effettua nel grafo 
HNSW della stessa categoria del metadato della query. I parametri di ricerca nell'HNSW
sono i seguenti 
$$K = 140 \qquad M= \lceil1800+(700)/(maxc\_size - minc\_size)dim\_cat\rceil$$
Dove:
\begin{itemize}
    \item $maxc\_size$: è la dimensione della categoria più grande presente 
    nel dataset.
    \item $minc\_size$: dimensione della categoria più piccola presente.
    nel dataset
    \item $dim\_cat$: dimensione della categoria in cui afferisce la query.
\end{itemize}
Dal risultato estratto dal grafo si effettua un raffinamento andando ad estrarre 
i $100$ migliori vettori nello spazio originale.

La \textbf{terza strategia} di ricerca è per le query di tipo $3$ che ha una selettività 
maggiore di $800000$ vettori. In questo caso la strategia è simile alla seconda 
ma leggermente differente, non si visitano i nodi che hanno timestamp fuori dall'intervallo 7
di query. Questo è reso possibile perché il grafo è fortemente connesso quindi 
si riesce a visitare i vicini di un nodo escluso precedentemente attraverso un altro 
nodo che ha timestamp all'interno dell'intervallo di query.

$$K = 150 \qquad M= \lceil1800+(1000)/(maxc\_size - minc\_size)dim\_cat\rceil$$

La \textbf{quarta strategia} di ricerca è per le query di tipo $0$. In questo caso 
si effettua una ricerca parallela nei $10$ HNSW dei sotto-intervalli di timestamp 
usando per ognuna i seguenti parametri:
$$M=450 \qquad K=150$$
In questo modo si ottengono $150\cdot 10$ vicini alla query nello spazio delle 
codifiche, successivamente si raffinano i risultati estraendo i migliori $100$ 
vettori rispetto allo spazio originale.

La \textbf{quinta strategia} di ricerca è per la query di tipo $2$ che ha una 
superiore a $500000$. In questa strategia si sfruttano gli HNSW creati per i $10$
sottointervalli di timestamp, più precisamente si controlla la copertura dell'intervallo 
$[l,r]$ della query rispetto ai sottointervalli $[l',r']$ e se $[l,r]\cap [l',r']$ 
allora si ricerca nell'HNSW del sottointervallo $[l',r']$. 
In particolare in base alla selettività della query variano i parametri di ricerca:
\begin{itemize}
    \item se la query ha una selettività inferiore a $2000000$ di vettori, allora 
    per un generico sottointervallo $[l',r']$ abbiamo le seguenti casistiche:
    \begin{itemize}
        \item $[l,r]\cap [l',r'] = \emptyset$: si ignora il sottointervallo perché 
        l'HNSW contiene solo vettori in $[l',r']$ che sono fuori dall'intervallo 
        di query.
        \item $[l,r] = [l',r'] \lor  [l',r'] \subseteq [l,r]$: tutti i vettori 
        del sottointervallo solo potenziali candidati, quindi si effettua una ricerca nell'HNSW
        utilizzando parametri $M=780$ e $K=150$.
        \item $[l',r'] \cap [l,r] \ne \emptyset, [l',r'] \not \subseteq [l,r]$: 
        solo una parte dei vettori sono dei potenziali candidati, quindi in base 
        all'ampiezza di $[l',r'] \cap [l,r]$ si hanno diverse strategie:
        \begin{itemize}
            \item se l'intersezione è inferiore al $50\%$ di $[l',r']$ allora si effettua una 
            ricerca esaustiva delle $150$ codifiche più vicine alla codifica del 
            vettore di query. Per fare ciò si sfrutta la $t\_map$ e 
            l'ordinamento delle codifiche rispetto il timestamp.
            \item se l'intersezione è superiore al $50\%$ di $[l',r']$ allora si effettua una 
            ricerca nell'HNSW del sottointervallo $[l',r']$ con i parametri $M=1180$ e $K=150$.
        \end{itemize}
    \end{itemize}
    \item se la query ha una selettività superiore a $2000000$ di vettori, allora 
    per un generico sottointervallo $[l',r']$ abbiamo le stesse casistiche 
    del caso con selettività inferiore a $2000000$, con la differenza che cambia
    la soglia per la ricerca esaustiva e i parametri della ricerca nel grafo. Più 
    precisamente:
    \begin{itemize}
        \item se $[l',r'] \not \subseteq [l,r]$ allora si ignora il sottointervallo.
        \item se $[l,r] = [l',r'] \lor  [l',r'] \subseteq [l,r]$:
        \begin{itemize}
            \item se la query seleziona fino a $3000000$ vettori allora ricerca 
            nell'HNSW di $[l',r']$ con $K=150$ e $M=780$
            \item se la query seleziona da $3000000$ a $6000000$ vettori allora ricerca 
            nell'HNSW di $[l',r']$ con $K=150$ e $M=630$
            \item se la query seleziona sopra $6000000$ vettori allora ricerca 
            nell'HNSW di $[l',r']$ con $K=150$ e $M=480$
        \end{itemize}
        \item se $[l',r'] \cap [l,r] \ne \emptyset, [l',r'] \not \subseteq [l,r]$:
        \begin{itemize}
            \item se l'intersezione è inferiore al $20\%$ di $[l',r']$ allora si effettua una 
            ricerca esaustiva delle $150$ codifiche più vicine alla codifica del 
            vettore di query. Per fare ciò si sfrutta la $t\_map$ e 
            l'ordinamento delle codifiche rispetto il timestamp.
            \item se l'intersezione è superiore al $20\%$ di $[l',r']$ allora si effettua una 
            ricerca nell'HNSW del sottointervallo $[l',r']$ usando i seguenti parametri:
            \begin{itemize}
                \item se la query seleziona fino a $3000000$ vettori allora ricerca 
                nell'HNSW di $[l',r']$ con $K=150$ e $M=1180$
                \item se la query seleziona da $3000000$ a $6000000$ vettori allora ricerca 
                nell'HNSW di $[l',r']$ con $K=150$ e $M=780$
                \item se la query seleziona sopra $6000000$ vettori allora ricerca 
                nell'HNSW di $[l',r']$ con $K=150$ e $M=680$
            \end{itemize}
        \end{itemize}
    \end{itemize}
\end{itemize}

Quindi alla fine si producono un totale di $150$ vettori per ogni sottointervallo
$[l',r']$ con intersezione non nulla con l'intervallo di query e, in conclusione, 
si raffina il risultato estraendo i migliori $100$ vettori più vicini alla query 
nello spazio originale.

\subsection{Considerazioni sulla ricerca}

Dal momento che sono state definite diverse strategie di ricerca nidificate
\`e piuttosto complicato la sua analisi di complessità di ricerca di una query,
quindi il risultato mostrato in questa sezione \`e da intendersi come qualitativo.

L'osservazione più naturale che nasce dopo aver visto per la prima volta la ricerca
è che la quantizzazione è l'elemento chiave. Infatti calcolare la distanza sugli 
elementi quantizzazioni risulta nettamente più efficiente rispetto a calcolare la 
distanza sugli elementi, questo permette quindi agli autori di effettuare ricerche 
esaustive su numerosi elementi, cosa che noi non potevamo permetterci di fare.

Inoltre, la ricerca consiste alla fine nell'effettuare una ricerca esausiva 
oppure nell'effettuare una ricerca nell'HNSW. In generale la ricerca esaustiva è 
lineare ma limitata a massimo $1000000$ di vettori (nel caso della ricerca di una 
query di tipo 2), difficile che possa capitare dal momento che sarebbero tutti 
all'interno di un sottointervallo. Al contrario la ricerca nell'HNSW si effettua 
in tempo logaritmico e questo permette di velocizzare le ricerche per query ad 
alta selettività, in aggiunta mentre si effettua la ricerca nel grafo spesso vengono 
applicata delle ottimizzazioni per scartare alcuni nodi evitando di calcolare 
ulteriori distanze. 

Alla luce di queste osservazioni si deriva che per una singola query
nel caso migliore abbiamo una complessit\`a di $\mathcal{O}(\log |D|)$,
mentre nel caso peggiore si riduce ad un semplice tempo sublineare
$\mathcal{O}(|D|)$. Nonostante il tempo complessivo scali al massimo come
$\mathcal{O}(|D| \times |Q|)$, se si assume una distribuzione uniforme del tipo di query
e della dimensione della categoria e del range di timestamp, l'algoritmo usa una ricerca
logaritmica per pi\`u della met\`a del tempo. Se si aggiunge l'uso della quantizzazione
simmetrica e della parallelizzazione l'approccio presentato raggiunge velocit\`a impressionanti
mantenendo al contempo un'eccellente accuratezza (pari al 100\%, dalla piattaforma Sigmod).

%Tempi di ricerca:
% - Per query di tipo 0
%   Log(|D|)
% - Per query di tipo 1
%   <  500000 -> O(\frac {|D|} {20}) -> D/20 = 500000
%   >= 500000 -> Log(|D|)
% - Per query di tipo 2
%   <  500000 -> O(\frac {|D|} {20}) -> D/20 = 500000
%   >= 500000 -> Log(|D|)
% - Per query di tipo 3
%   <  500000 -> O(\frac {|D|} {12.5}) -> D/12.5 = 800000
%   >= 500000 -> Log(|D|)
