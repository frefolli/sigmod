\chapter{Soluzione vincente}
La soluzione vincente si basa su HSNW, più precisamente viene costruita una 
struttura di indicizzazione ibrida composta da un HSNW per delle

\section{Creazione della struttura di indicizzazione}
Per prima cosa vene letto tutto il dataset e viene salvato
all'interno di $3$ array:
\begin{itemize}
    \item base\_vecs: array contenente i coefficienti di tutti i vettori del dataset
    di dimensioen $10^7\cdot 100$.
    \item base\_labels: array contente le labels di tutti i vettori, ovvero un 
    vettore di $10^7$ elementi.
    \item base\_labels: array contente le labels di tutti i vettori, ovvero un 
    vettore di $10^7$ elementi.
\end{itemize} 

La posizione $i$ indica $i$-esimo vettore nel dataset.

Successivamente si creano degli array di appoggio che specificano l'ordinamento 
del dataset rispetto a diversi criteri di ordinamento:
\begin{itemize}
    \item sorted\_base\_ids: vettore contenente gli indici dei vettori ordinati 
    per categoria crescente e successivamente per timestamps. Quindi per ogni $i\le j\in \mathbb{N}$,
    allora il numero della categoria associato al vettore $sorted\_base\_ids[i]$
    è minore rispetto al numero della categoria del vettore $sorted\_base\_ids[j]$,
    in caso di categorie uguali allora si ordina per timestamps.
    \item sorted\_base\_ids\_by\_time: vettore contenente gli indici dei vettori ordinati 
    per timestamps spostando i spostando tutti i vettori della categoria più grande all'inizio 
    del vettore. In questo modo all'inizio del vettore si avranno solo vettori 
    ordinati per timestamps della categoria più grande, successivamente si hanno 
    tutti gli altri vettori ordinati per timestamp. 
    \item sorted\_base\_ids\_by\_full\_time: vettore contenente gli indici dei vettori ordinati 
    per timestamps.
\end{itemize}

\begin{nota}
    L'ordinamento dei vettori della categoria più grande in sorted\_base\_ids\_by\_time 
    viene replicato dal vettore sorted\_base\_ids perché in caso ci fossero vettori 
    della categoria più grande con lo stesso timestap allora questi possono avere 
    un ordinamento diverso tra i due indici.
\end{nota}

In aggiunta viene creata una $c\_map$, ovvero una mappa delle categorie ovvero 
un array di coppie ordinate:
\begin{equation}
    category\_map[c] = \langle i, dimensione\_categoria\_c\rangle
\end{equation} 
dove $i = sorted\_base\_ids[v]$ con $v$ ultimo vettore della categoria $c$, 
mentre $dimensione\_categoria\_c$ specifica il numero di vettori presenti nella 
categoria $c$. Da notare che la coppia viene aggiunta all'array solo quando $c$ 
ha almeno $450000$ vettori.

Successivamente si costruisce una $t\_map$, ovvero una mappa dei timestamp ovvero 
un array di coppie ordinate:
\begin{equation}
    t\_map[i] = \langle l, dimensione\_intervallo\rangle
\end{equation} 
dove $i$ specifica l'intervallo di timestamp $[0.1*i, 0.1*(i+1)]$, $l= timestamps\_by\_full\_time[v]$
dove $v$ è il primo vettore con il timestamp nell'intervallo cercato, mentre 
$dimensione\_intervallo$ il numero di vettori nell'intervallo.

In seguito viene letto il query set e viene salvato all'interno di due array:
\begin{itemize}
    \item query\_vecs: array contente le componenti dei vettori di query ovvero 
    $10^6 \cdot 100$.
    \item query\_metas: array contente i metadati delle query, ovvero la tipologia,
    la categoria e l'intervallo di timestamp.
    $10^6 \cdot 4$.
\end{itemize}

Successivamente si creano degli array di appoggio:
\begin{itemize}
    \item sorted\_ids: vettore degli indici dei delle query ordinate per tipo, 
    categoria, lower bound e upper bound dell'intervallo di timestamp
    \item category\_query: mappa che associa per ogni categoria un vettore di 
    indici di query che afferiscono a quella categoria.
\end{itemize}

Da notare che la $c\_map$, $category\_query$ e $t\_map$ sono tutte mappe implementate 
come \texttt{unordered\_map}.

Dopo la lettura ordinata del dataset e del queryset, è stata creata la struttura 
di indicizzazione sul dataset. Più precisamente è stato creato un HSNW per i vettori 
afferenti alla categoria più grande, uno per i vettori afferenti a categorie che 
contengono almento $500000$ vettori e uno per ogni intervallo di timestamp della 
$t\_map$. 

\subsection{Creazione di HNSW}
Sfruttano l'implementazione di HNSW della libreria \href{https://github.com/zilliztech/pyglass/tree/master}{pyglass}
unito ad un processo di quantizzazione dei vettori.

HNSW è una struttura dati basata su grafi metrici a livelli che permette di navigare
lo spazio dei vettori navigado attraverso vettori vicini. Ogni livello è una 
rappresentazione dello spazio gentrico tramite un grafo metrico. Nel grafo ogni nodo rappresenta 
un vettore e si hanno degli archi diretti che rappresentano la relazione di vicinanza,
ovvero, dato un nodo ogni suo arco uscente porta ai $M$ nodi più vicini di quel nodo. La relazione 
di vicinanza è diretta perché nel caso dell'arco $(a,b)$, non è detto che il nodo 
$b$ abbia nei $M$ nodi più vicini il nodo $a$.

Ogni livello rappresenta il livello di approssimazione dello spazio geometrico, 
più il livello è alto allora maggiore è l'approssimazione, più il livello è basso 
e più precisa sarà la rappresentazione dello spazio. Dato un livello $l$, il livello 
successivo $l+1$ è un grafo composto da almeno gli stessi nodi del grafo sul livello 
precedente con l'aggiunta di più nodi vicini. Questo permette di navigare 
lo spazio velocemente nel livello più alto non passando per tutti i punti perché 
ogni nodo di un livello rappresenta una \textbf{Voronoi Cell} dello spazio grande 
quanto la specificità del livello. 

Dato un insieme di vettori $V$, la costruzione di un generico HNSW  è un processo 
iterativo, in cui si inserisce ciascun vettore uno alla volta. L'inserimento di 
un vettore $q\in V$ nella struttura si articola nelle seguenti fasi:
\begin{itemize}
    \item \textbf{inizializzazione}: si seglie un nodo $ep$ di entry point nel livello massimo
    e si seleziona il livello $l$ a partire dal quale si inserice il vettore $v$.
    Più precisamente $l = \lfloor-\ln (U[0,1])\cdot m_l\rfloor$ e si inserirà $q$ 
    in tutti i livelli da $l$ a $0$. $m_l$ è la costante di normalizzazione che 
    limita $l\in [0, L_{max}]$, dove $L_{max}$ è il livello massimo della struttura.
    \item \textbf{prima fase}: a partire dal livello $L_{max}$, si naviga il grafo 
    a partire dal nodo $ep$ fino a quando non si trova il nodo $w$ più vicino a
    $q$ (algoritmo greedy), successivamente si effettua la stessa operazione 
    su livello $L_{max} - 1$ utilizzando come nodo di entry point il nodo $w$ fino 
    a quando non si trova l'entry point $w$ del livello $l$.
    \item \textbf{seconda fase}: a partire dal livello $l$ si cercano i $W$ nodi 
    più vicini a $q$ a partire dall'ultimo entry point trovato al termine della fase precedente ($|W| = efConstruction$).
    Successivamente si usa un'euristica che calcola dall'insieme dei $W$ nodi
    l'insieme $N$ degli $M$ nodi più vicini a $q$ sempre sullo stesso layer $l$. 
    Si creano degli edge bidirezionali tra i nodi $N$ e $q$. Successivamente si 
    sistemano i collegamenti tra i vicini in modo che ogni nodo rispetti il vincolo 
    di avere al massimo $M_{max}$ collegamenti, questo viene implementato rimuovendo 
    degli edge tra due nodi nel vicinato di $q$. Infine si reitera il procedimento 
    dal livello $l$ al livello $0$.
\end{itemize}

Per la costruzione della struttura si richiedono i seguenti parametri:
\begin{itemize}
    \item $M_{max}$: numero massimo di edge uscenti da ogni nodo per un livello, questo 
    potrebbe coincidere con $M$.
    \item $L_{max}$: livello massimo della struttura.
    \item $efConstruction$: specifica il numero dei nodi candidati ad essere i vicini 
    del vettore $v$. Questo parametro permette di variare la recall e il tempo di 
    costruzione della struttura.
\end{itemize} 

La libreria implementa la costruzione scegliendo dei nodi di entry point dal livello 
massimo in modo randomico e l'inserimento dei nodi viene parallelizzato sfruttando 
i lock. In aggiunta la libreria non lavora direttamente sui vettori nativi, bensì
applica una quantizzazione scalare simmetrica a $8$ bit in modo tale da ridurre la dimensione 
dei vettori a $112 B$ e sfruttare le operazioni vettoriali del processore per velocizzare 
il calcolo delle distanze.

I parametri di costruzione specificati dagli autori sono $M = 28$ e $efConstruction = 200$.

% TODO: QUANTI LIVELLI HA IL GRAFO? forse 1/ln(M)
% TODO: COME LO COSTRUISCE PARALLELAMENTE?

\subsection{Ricerca nella struttura}
La ricerca di una query $q$ in un generico HNSW richiede il parametro $K$ che specifica i numero 
di vicini alla query, il parametro $ef$ che è il numero di candidati vicini da considerare.

La ricerca si articola nel scegliere un entry point nel livello massimo randomico, 
si cerca il miglio vicino nel livello e lo si usa come entry point per il livello 
successivo. Si reitera fino a quando si trova l'entry point del livello $0$. 

Infine sul livello $0$ si cerca $ef$ migliori vicini e si estraggono i migliori $K$.
\section{Ricerca nella struttura di indicizzazione}

La ricerca 

- selezionata i vettori codificati in base alla tipologia di query
- se < 500000 oppure query tipo 3 < 800000 allora fa bruteforce e ricerca le migliori $140$ codifiche a $112 B$
più vicine a quella di query e poi seleziona le $100$ migliori rispetto ai valori 
nativi
- per tipo 1 > 500000 e tipo 3 > 800000  effetua una ricerca in HNSW $K= \lceil1800+(2500-1800)/(maxc_size - minc_size)dim_cat\rceil$
 $K= \lceil1800+(2800-1800)/(maxc_size - minc_size)dim_cat\rceil$
 Seleziona i migliori $150$ vettori che rispettano il filtro, infine estraggo i 
 migliori $100$.

- per tipo 0 oltre alla ricerca in brute force, ricerca nei grafi dei timestamp.
 Più precisamente si ottengono $150$ candidati per ciascuna query dalla ricerca 
 parallela nei grafi di ciascun timestamp. Più precisamente per ciascun grafo di 
 timestamp si cercano $450$ vicini alla query, dai quali si estraggono solo i primi 
 $150$ per ogni grafo e, infine, si aggregano i risultati in un pool 
 di candidati per ogni query grande $150$ vettori preferendo sempre i più vicini.
 (nota nella parte di bruteforce cerca un solo vicino, forse è un errore che hanno 
 lasciato)

 Avendo $10$ sottointervalli di timestamp si ottengono un totale di $150*10$ candidati 
 e si aggregano scegliendo i migliori $100$.
 
- tipo 2: se quelli selezionati rispetto al filtro sono > 500000 e > 800000 allora 
 seleziona la strategia di ricerca in base alla copertura di $[l,r]$ sulle $10$ 
 suddivisioni del timestamp, data una suddivisione $[l',r']$:
    - se il numero di vettori selezionati in precedenza è < 2000000 e $[l,r]$ copre 
    tutto $[l',r']$ allora si cerca direttamente nel grafo metrico dell'intervallo $[l',r']$ (ricerca FULL),
    se [l,r] copre fino ad un massimo del $50\%$ della suddivisione [l',r'] allora 
    ricerca in bruteforce in quell'intervallo. Altrimenti fai ricerca MEDIUM
    - se il numero di vettori selezionati in precedenza è >= 2000000 fai la stessa 
    cosa di sopra ma abbassando la threshold di bruteforce a  $20\%$

  Effettua la ricerca della query rispetto alle strategie definite prima:
    - bruteforce: i migliori 140 vettori nel sottointervallo [l',r'] intersecato [l,r]
    e li mette nel pool di candidati.
    - MEDIUM: se l'insieme dei vettori da confrontare è < 3000000 allora si estraggono 
    dal sottointervallo [l',r'] i vicini $1180$ dalla query, se sono in totale <= 6000000 e >=  3000000
    allora si estraggono dal sottointervallo [l',r'] i vicini $780$ dalla query, 
    se sono in totale > 6000000 allora si estraggono dal sottointervallo [l',r'] i vicini $680$ dalla query.
    e di questi se ne estraggono $150$
    - FULL: se l'insieme dei vettori da confrontare è < 3000000 allora si estraggono 
    dal sottointervallo [l',r'] i vicini $780$ dalla query, se sono in totale <= 6000000 e >=  3000000
    allora si estraggono dal sottointervallo [l',r'] i vicini $630$ dalla query, 
    se sono in totale > 6000000 allora si estraggono dal sottointervallo [l',r'] i vicini $480$ dalla query.
    e di questi se ne estraggono $150$
  
  In questo modo se $[l,r]$ include in totale $5$ sotto intervalli contigui [l',r']
  allora si ottengono $150*5$ e infine si scelgono i migliori $100$.